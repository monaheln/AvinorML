\documentclass[11pt,a4paper]{article}
\usepackage[utf8]{inputenc}
\usepackage[norsk]{babel}
\usepackage{amsmath,amsfonts,amssymb}
\usepackage{graphicx}
\usepackage{booktabs}
\usepackage{url}
\usepackage{listings}
\usepackage{xcolor}
\usepackage{geometry}
\geometry{margin=2.5cm}

% Code styling
\lstset{
    language=Python,
    basicstyle=\ttfamily\small,
    keywordstyle=\color{blue},
    commentstyle=\color{gray},
    stringstyle=\color{red},
    numbers=left,
    numberstyle=\tiny,
    frame=single,
    breaklines=true
}

\title{AFIS Concurrency Prediction Model - Dokumentasjon for Datakonkurranse}
\author{Mona Helness}
\date{\today}

\begin{document}

\maketitle

\begin{abstract}
Denne rapporten dokumenterer løsningen for Avinors datakonkurranse "Når går det på høygir?". Løsningen predikerer sannsynlighet for samtidighetshendelser i AFIS-operasjoner ved bruk av Random Forest maskinlæring. Modellen oppnår AUC-ROC på 0.9563 og identifiserer antall planlagte fly og kommunikasjonsvindu-overlapp som de viktigste risikofaktorene for oktober 2025.
\end{abstract}

\section{a. Metodevalg og tilnærming}

\subsection{Problemstilling og tilnærming}
Oppgaven er å predikere sannsynlighet for samtidighetshendelser hvor AFIS-operatører kommuniserer med flere fly samtidig. Samtidighet defineres av overlappende kommunikasjonsintervaller:
\begin{itemize}
    \item \textbf{Ankomster}: 16 minutter før til 5 minutter etter landing
    \item \textbf{Avganger}: 15 minutter før til 8 minutter etter avgang
\end{itemize}

\subsection{Valgt metodikk}
\textbf{Random Forest maskinlæring} ble valgt som hovedmetode.

\textbf{Begrunnelse for metodevalgene}:
\begin{enumerate}
    \item \textbf{Binær klassifisering}: Problemet krever predikering av samtidighet/ikke-samtidighet per time
    \item \textbf{Forklarbarhet}: Avinor krever transparente modeller for regulatoriske formål
    \item \textbf{Mixed data types}: Metoden håndterer både kategoriske (flyplassgruppe, sesong) og numeriske (flyantall, tid) variabler
    \item \textbf{Robust mot outliers}: Viktig for operasjonelle data med potensielle datafeil
    \item \textbf{Feature importance}: Gir innsikt i hvilke faktorer som driver samtidighetsrisiko
    \item \textbf{Ubalanserte klasser}: Samtidighetshendelser er sjeldne - Random Forest håndterer dette godt
\end{enumerate}

\subsection{Datagrunnlag}
Løsningen baserer seg på:
\begin{itemize}
    \item Historiske flydata (2018-2025): 399,426 flybevegelser
    \item Eksisterende treningsdata: 465,031 time-observasjoner med targets
    \item Oktober 2025 ruteplan: 4,565 planlagte fly (oppdatert versjon)
    \item Oktober 2025 inferens-data: 5,208 time-observasjoner med pre-beregnede features
    \item 7 flyplassgrupper (A-G) med 3 flyplasser hver
\end{itemize}

\textbf{Oppdaterte datasett}: Modellen bruker de nyeste versjonene av oktober-dataene (\texttt{schedule\_oct2025\_updated.csv} og \texttt{inference\_data\_oct2025\_updated.csv}) som inneholder flere fly og korrigerte features sammenlignet med originalversjonene.

\section{b. Systemstruktur og arkitektur}

\subsection{Systemarkitektur og komponenter}

\textbf{Overordnet arkitektur}:
\begin{center}
\texttt{CSV Data} $\rightarrow$ \texttt{Python Databehandling} $\rightarrow$ \texttt{Scikit-learn ML} $\rightarrow$ \texttt{CSV Prediksjoner}
\end{center}

\textbf{Tekniske komponenter}:
\begin{itemize}
    \item \textbf{Datainnlesing}: Pandas for CSV-håndtering
    \item \textbf{Feature Engineering}: Custom Python-funksjoner
    \item \textbf{Maskinlæring}: Scikit-learn RandomForestClassifier
    \item \textbf{Validering}: TimeSeriesSplit for tidsbasert validering
    \item \textbf{Output}: CSV-generator for kompetanse-formatet
\end{itemize}

\textbf{Ingen eksterne API-er}: Løsningen er standalone og krever kun Python-biblioteker.

\subsection{Dataflyt og prosessering}

\textbf{Steg 1}: Innlesing av treningsdata (465,031 timer), oktober-ruteplan (4,565 fly) og inferens-data (5,208 timer)

\textbf{Steg 2}: Feature engineering - transformasjon av rådata til prediksjonsvariabler

\textbf{Steg 3}: Tidsbasert datasplit (før 2024 = trening, 2024+ = validering)

\textbf{Steg 4}: Random Forest trening med hyperparameter-optimalisering

\textbf{Steg 5}: Generering av oktober 2025 prediksjoner (5,047 timer)

\subsection{Arkitektur-visualisering}
\begin{center}
\begin{tabular}{|c|c|c|c|}
\hline
\textbf{Input} & \textbf{Processing} & \textbf{Model} & \textbf{Output} \\
\hline
Flydata CSV & Feature Engineering & Random Forest & Prediksjoner CSV \\
Treningsdata CSV & Tidsbasert Split & 100 Trees & 5,047 timer \\
Oktober Ruteplan & LabelEncoder & AUC=0.956 & Sannsynligheter \\
\hline
\end{tabular}
\end{center}

\section{c. Modeller og algoritmer}

\subsection{Hovedmodell: Random Forest Classifier}

\textbf{Algoritme}: RandomForestClassifier fra scikit-learn

\textbf{Parametervalg}:
\begin{itemize}
    \item \texttt{n\_estimators=100}: 100 beslutningstrær i ensemblet
    \item \texttt{max\_depth=10}: Maksimal tredybde for å unngå overfitting
    \item \texttt{min\_samples\_split=10}: Minimum samples for å dele en node
    \item \texttt{class\_weight='balanced'}: Håndterer ubalanserte klasser automatisk
    \item \texttt{random\_state=42}: For reproduserbare resultater
\end{itemize}

\subsection{Feature engineering}

\textbf{Tidsbaserte variabler}:
\begin{itemize}
    \item \texttt{hour\_of\_day}: Time på døgnet (0-23)
    \item \texttt{day\_of\_week}: Ukedag (0-6)
    \item \texttt{month}: Måned (1-12)
    \item \texttt{is\_weekend}: Binær helg-indikator
    \item \texttt{hour\_group}: Kategorisk (night/morning/afternoon/evening)
\end{itemize}

\textbf{Operasjonelle variabler}:
\begin{itemize}
    \item \texttt{feat\_sched\_flights\_cnt}: Antall planlagte fly per time
    \item \texttt{feat\_sched\_concurrence}: Eksisterende samtidighets-indikator
    \item \texttt{concurrence\_risk}: Binær flagg for timer med >1 fly
    \item \texttt{flights\_x\_hour}: Interaksjon flyvolum × tidspunkt
\end{itemize}

\textbf{Kategorisk encoding}:
\begin{itemize}
    \item \texttt{airport\_group\_encoded}: LabelEncoder for flyplassgrupper
    \item \texttt{season\_encoded}: LabelEncoder for sesonger
    \item One-hot encoding for \texttt{hour\_group} kategorier
\end{itemize}

\subsection{Evalueringsmetoder}

\textbf{Tidsbasert validering}:
\begin{itemize}
    \item \textbf{Treningsdata}: Før 2024 (39,478 observasjoner)
    \item \textbf{Valideringsdata}: 2024+ (10,522 observasjoner)
    \item \textbf{Rationale}: Unngår data leakage ved å ikke bruke fremtidige data
\end{itemize}

\textbf{Evalueringsmetrikk}:
\begin{itemize}
    \item \textbf{Primær}: AUC-ROC (som spesifisert i konkurransekriteriene)
    \item \textbf{Sekundær}: Precision, Recall, F1-score for klassifikasjon
    \item \textbf{Oppnådd resultat}: AUC-ROC = 0.9563
\end{itemize}

\section{d. Kildekode}

\subsection{Oversikt over kodebasen}

\textbf{Hovedfiler}:
\begin{itemize}
    \item \texttt{afis\_simple\_model.py}: Hovedimplementering (funksjoner og kjører prediksjoner)
    \item \texttt{afis\_concurrency\_model.py}: Utvidet implementering med detaljert feature engineering
    \item \texttt{Helness\_october\_2025\_predictions.csv}: Ferdig prediksjonsfil for innlevering
    \item \texttt{feature\_importance.csv}: Modellforklaring og feature-ranking
\end{itemize}

\subsection{Hovedkomponenter i koden}

\textbf{SimplifiedAFISModel klasse} med følgende metoder:

\begin{lstlisting}[caption=Hovedstruktur]
class SimplifiedAFISModel:
    def load_and_prepare_data(self):
        # Laster CSV-filer med pandas
        # Renser data (fjerner kansellasjoner, håndterer datoer)

    def engineer_core_features(self, data_df):
        # Lager tidsbaserte features (time, ukedag, sesong)
        # Lager operasjonelle features (flyantall, samtidighet)
        # Encoder kategoriske variabler

    def train_model(self, features_df):
        # Tidsbasert datasplit (før/etter 2024)
        # Random Forest trening med Grid Search
        # Feature importance analyse

    def predict_october(self, model, feature_cols):
        # Lager oktober template med alle timer/grupper
        # Beregner features fra oktober-ruteplanen
        # Genererer sannsynligheter per time
\end{lstlisting}

\subsection{Installasjon og kjøring}

\textbf{Systemkrav}:
\begin{itemize}
    \item Python 3.9+
    \item pandas, scikit-learn, numpy (standard ML-biblioteker)
\end{itemize}

\textbf{Installasjonsinstruksjoner}:
\begin{lstlisting}[language=bash]
# Installer nødvendige pakker
pip install pandas scikit-learn numpy

# Kjør hovedmodellen
python afis_simple_model.py
\end{lstlisting}

\textbf{Output}:
\begin{itemize}
    \item \texttt{Helness\_october\_2025\_predictions.csv}: Hovedleveranse
    \item \texttt{feature\_importance.csv}: Modellforklaring
    \item Konsoll-output med AUC og modellstatistikk
\end{itemize}

\subsection{Videreutvikling og skalering}

\textbf{Skalering}:
\begin{itemize}
    \item Modellen kan håndtere større datasett ved å øke sample-størrelsen i \texttt{model.training\_data.sample()}
    \item Random Forest parallellisering gjennom \texttt{n\_jobs=-1} for raskere trening
\end{itemize}

\textbf{Utvidelsesmuligheter}:
\begin{itemize}
    \item \textbf{Værdata}: Integrer METAR API for vær-features
    \item \textbf{Real-time}: Koble til live flydata for sanntids-prediksjoner
    \item \textbf{Andre algoritmer}: XGBoost implementering finnes i \texttt{afis\_concurrency\_model.py}
    \item \textbf{Feature engineering}: \texttt{afis\_concurrency\_model.py} viser avanserte features som kommunikasjonsvindu-beregning
\end{itemize}

\textbf{Vedlikehold}:
\begin{itemize}
    \item Modellen kan retrenes automatisk med nye data uten kodeendringer
    \item Feature engineering er modulært og kan utvides
    \item Tidsbasert validering sikrer realistisk evaluering av nye modeller
\end{itemize}

\subsection{Hovedresultater}

\textbf{Modellytelse}:
\begin{itemize}
    \item \textbf{AUC-ROC}: 0.9563 (utmerket diskrimineringsevne)
    \item \textbf{Treningssett}: 39,478 observasjoner, positive rate 21.6\%
    \item \textbf{Valideringssett}: 10,522 observasjoner, positive rate 25.2\%
\end{itemize}

\textbf{Oktober 2025 prediksjoner}:
\begin{itemize}
    \item \textbf{Totalt}: 5,047 time-prediksjoner generert
    \item \textbf{Gjennomsnittlig risiko}: 27.85\%
    \item \textbf{Høyrisiko-timer} ($>0.5$): 1,256 timer (24.9\%)
    \item \textbf{Sannsynlighetsområde}: 0.002 - 0.991
\end{itemize}

\textbf{Viktigste risikofaktorer}:
\begin{center}
\begin{tabular}{lr}
\toprule
Feature & Importance \\
\midrule
Planlagt samtidighet & 34.0\% \\
Antall fly per time & 20.1\% \\
Samtidighets-risiko flagg & 19.7\% \\
Fly × time interaksjon & 13.9\% \\
Flyplassgruppe & 3.9\% \\
\bottomrule
\end{tabular}
\end{center}

\subsection{Konklusjon}

Løsningen oppfyller alle krav i datakonkurransen og leverer:

\begin{enumerate}
    \item \textbf{Høy nøyaktighet}: 95.6\% AUC overgår forventede standarder
    \item \textbf{Praktisk anvendelighet}: Identifiserer 1,256 høyrisiko-timer for oktober
    \item \textbf{Forklarbarhet}: Klare innsikter i at flyvolum driver samtidighetsrisiko
    \item \textbf{Operasjonell verdi}: Gir handlingsrettede innsikter for bemanningsplanlegging
\end{enumerate}

Modellen viser at timer med $\geq 2$ planlagte fly har dramatisk høyere samtidighetsrisiko, noe som bekrefter operasjonell erfaring og gir Avinor datagrunnlag for evidensbaserte bemanningsbeslutninger.

\end{document}